
\documentclass[conference]{IEEEtran}


% If IEEEtran.cls has not been installed into the LaTeX system files,
% manually specify the path to it like:
% \documentclass[conference]{../sty/IEEEtran}


% Some very useful LaTeX packages include:
% (uncomment the ones you want to load)

\usepackage[letterpaper, left=1in, right=1in, bottom=1in, top=0.75in]{geometry}
%\usepackage{fancyhdr}
%\usepackage{cite}
%\usepackage{graphicx}
%\usepackage{psfrag}
%\usepackage{subfigure}
%\usepackage{url}
%\usepackage{stfloats}
%\usepackage{amsmath}
%\usepackage{array}
%\usepackage{fancyhdr}
%\usepackage{epsfig}
%\usepackage{amssymb}
%\usepackage{color}



\begin{document}
%
% paper title
% Titles are generally capitalized except for words such as a, an, and, as,
% at, but, by, for, in, nor, of, on, or, the, to and up, which are usually
% not capitalized unless they are the first or last word of the title.
% Linebreaks \\ can be used within to get better formatting as desired.
% Do not put math or special symbols in the title.
\title{SONOHI: A modular simulation framework with integrated SDR for producing synthetic datasets}


% author names and affiliations
% use a multiple column layout for up to three different
% affiliations
\author{\IEEEauthorblockN{First Last}
\IEEEauthorblockA{School of Electrical and\\Computer Engineering\\
University of Abcdefg\\
Email: email@domain.cedu}
\and
\IEEEauthorblockN{First Last}
\IEEEauthorblockA{School of Electrical and\\Computer Engineering\\
	University of Abcdefg\\
	Email: email@domain.cedu}
\and
\IEEEauthorblockN{First Last}
\IEEEauthorblockA{School of Electrical and\\Computer Engineering\\
	University of Abcdefg\\
	Email: email@domain.cedu}}

% conference papers do not typically use \thanks and this command
% is locked out in conference mode. If really needed, such as for
% the acknowledgment of grants, issue a \IEEEoverridecommandlockouts
% after \documentclass

% for over three affiliations, or if they all won't fit within the width
% of the page, use this alternative format:
%
%\author{\IEEEauthorblockN{Michael Shell\IEEEauthorrefmark{1},
%Homer Simpson\IEEEauthorrefmark{2},
%James Kirk\IEEEauthorrefmark{3},
%Montgomery Scott\IEEEauthorrefmark{3} and
%Eldon Tyrell\IEEEauthorrefmark{4}}
%\IEEEauthorblockA{\IEEEauthorrefmark{1}School of Electrical and Computer Engineering\\
%Georgia Institute of Technology,
%Atlanta, Georgia 30332--0250\\ Email: see http://www.michaelshell.org/contact.html}
%\IEEEauthorblockA{\IEEEauthorrefmark{2}Twentieth Century Fox, Springfield, USA\\
%Email: homer@thesimpsons.com}
%\IEEEauthorblockA{\IEEEauthorrefmark{3}Starfleet Academy, San Francisco, California 96678-2391\\
%Telephone: (800) 555--1212, Fax: (888) 555--1212}
%\IEEEauthorblockA{\IEEEauthorrefmark{4}Tyrell Inc., 123 Replicant Street, Los Angeles, California 90210--4321}}




% use for special paper notices
%\IEEEspecialpapernotice{(Invited Paper)}




% make the title area
\maketitle

% As a general rule, do not put math, special symbols or citations
% in the abstract
\begin{abstract}

\end{abstract}

% no keywords




% For peer review papers, you can put extra information on the cover
% page as needed:
% \ifCLASSOPTIONpeerreview
% \begin{center} \bfseries EDICS Category: 3-BBND \end{center}
% \fi
%
% For peerreview papers, this IEEEtran command inserts a page break and
% creates the second title. It will be ignored for other modes.
\IEEEpeerreviewmaketitle



\section{Introduction}
% Problem statement,
% 5G requirements, high SE, low latency. Overall more capacity
% Current 5G solutions have severe problems in network management and power consumption.
Next-generation mobile networks (5G) are required to offer increasingly high SE (spectral efficiency) and low latency to meet the demands and the drive for additional mobile services. [??] For the 5G related solutions to be viable, and be able to offer such high capacity networks in an effective manner solutions for the most pressing issues, 1) Interference and 2) Power consumption.

Add section about how these are expected/shown to be solved and where work is required.

It has been shown that learning techniques offer efficent solutions which are both adaptive, scalable and most importantly, applicable. The vast majority of the solutions take use of supervised learning techniques [??][??] where datasets are required for training. Some report the use of reinforcement learning techniques that gather data live and take appropriate actions that benefit the network or a subset of the network. In both cases getting access to realistic data from live networks is a cumbersome task.

In this paper we presents the first results of a fully modular Open-source mobile network framework implemented in MATLAB. The main purpose of which is to produce realistic datasets that can offer sufficent knowledge for training such learning techniques, and subsequently be applied successfully in live networks. The advantage is to avoid costly network measurements that might be needed for developing techniques that offer solutions to next-generation mobile networks. The first public version will include integration of State of the art traffic[??] and channel[??] models which are based on pedestrian mobility in Manhattan-like grids.  Because of the modularity, it is intended that the SONOHI framework is extended with vechicular movement and any type of urban environment.

\section{Framework structure}
The framework is organized as modular blocks. It is based on a base scenario which is given as a map in 3D. Placement of the basestations (Macro, Micro) can be either random or static.

\section{Traffic models}

\section{Channel models}
eHATA, WINNER II.

\section{Software-defined Radio}
Results and verification

\section{Conclusion}




% conference papers do not normally have an appendix


% use section* for acknowledgment
\section*{Acknowledgment}


The authors would like to thank...





% trigger a \newpage just before the given reference
% number - used to balance the columns on the last page
% adjust value as needed - may need to be readjusted if
% the document is modified later
%\IEEEtriggeratref{8}
% The "triggered" command can be changed if desired:
%\IEEEtriggercmd{\enlargethispage{-5in}}

% references section

% can use a bibliography generated by BibTeX as a .bbl file
% BibTeX documentation can be easily obtained at:
% http://mirror.ctan.org/biblio/bibtex/contrib/doc/
% The IEEEtran BibTeX style support page is at:
% http://www.michaelshell.org/tex/ieeetran/bibtex/
%\bibliographystyle{IEEEtran}
% argument is your BibTeX string definitions and bibliography database(s)
%\bibliography{IEEEabrv,../bib/paper}
%
% <OR> manually copy in the resultant .bbl file
% set second argument of \begin to the number of references
% (used to reserve space for the reference number labels box)
\begin{thebibliography}{1}

\bibitem{IEEEhowto:kopka}
H.~Kopka and P.~W. Daly, \emph{A Guide to \LaTeX}, 3rd~ed.\hskip 1em plus
  0.5em minus 0.4em\relax Harlow, England: Addison-Wesley, 1999.

\end{thebibliography}




% that's all folks
\end{document}
